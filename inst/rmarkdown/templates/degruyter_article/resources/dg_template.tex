%% This file is to be used as a template for your submission. 
%% Rename this file and replace the text with the text of 
%% your manuscript.
%%
%% The standard LaTeX document class "article" is recommended. 
%% Use options letterpaper and 12pt.
\documentclass[letterpaper,12pt]{article}

%% This is the recommended preamble for your document.

%% Load De Gruyter specific settings 
\usepackage{dgjournal}          

%% The mathptmx package is recommended for Times compatible math symbols.
%% Use mtpro2 or mathtime instead of mathptmx if you have the commercially
%% available MathTime fonts.
%% Other options are txfonts (free) or belleek (free) or TM-Math (commercial)
\usepackage{mathptmx}

%% Use the graphics package to include figures
\usepackage{graphics}

%% Use natbib with these recommended options
\usepackage[authoryear,comma,longnamesfirst,sectionbib]{natbib} 

%% Start your document body here
\begin{document}

%% Do NOT include any fronmatter information; including the title, author names,
%% institutes, acknowledgments and title footnotes (author information, funding
%% sources, etc.). Start the document with the first section or paragraph of
%% the article.

\section{This should be the first section of your document}

Your text goes here. Start with the first section or paragraph of your
article. Do not set use either headers or footers and do not set any running
heads or change any page numbers.

The title page along with headers and footers will be inserted by the EdiKit
system. Use the ``revise manuscript'' link to enter this information in the
EdiKit system.

Use the standard LaTeX commands to set the text of the article. The dgjournal
package works with any of the standard LaTeX document classes - article, report
or book. Other document classes should work but compliance with De Gruyter
requirements is not assured. For more information on LaTeX, see
\cite{lamport,mittelbach,oetiker}. \cite{mittelbach} is highly recommended.

Use the standard LaTeX sectioning commands for your headings.

\subsection{A second order heading}

Some text under the subheading. Paragraphs that follow heads are not
indented.

Math should also be set in Times. Use the mathptmx package if you do not have
any of the commercially available fonts that are compatible with Times.
\begin{equation}
    y^{(n)} = \sum_{i=0}^{n-1} a_i(x) y^{(i)} + r(x) 
\end{equation}

All environments provided by the standard LaTeX document classes are
unchanged. Vertical spaces within lists have been altered to comply with De Gruyter
requirements.
\begin{enumerate}
\item This is the first item within the list. Some more text here in order to
  display the alignment.
\item Another item in the list.
\item Yet another item in the list.
\end{enumerate}

Here is an example of a Figure. It's the same as in standard LaTeX.

\begin{figure}[!h]
%% Use the graphics package to insert figures
%% \includegraphics{figure.eps}
% Use \centering to center the table
\centering
%% A small box in place of a figure
\framebox{%
  \begin{minipage}{10pc}
    \begin{center}
      \vspace{1cm}\par
      A figure\par
      \vspace{1cm}
    \end{center}
  \end{minipage}}
\caption{Insert your caption here. If you wish to label your figure for
  cross-referencing, use a label either within the caption or after it.}
\label{fig1}
\end{figure}

An example of a table follows. This is also the same as in standard LaTeX.

\begin{table}[!h]
% Use \centering to center the table
\centering
\caption{Insert your table caption here. If you wish to label the table for
  cross-referencing, use a label either within the caption or after it.}
\begin{tabular}{llll}
\hline
Symbol        & LaTeX Command      & Symbol      & LaTeX Command \\
\hline
$\alpha$      & \verb+\alpha+      & $\zeta$     & \verb+\zeta+ \\
$\beta$       & \verb+\beta+       & $\eta$      & \verb+\eta+ \\
$\gamma$      & \verb+\gamma+      & $\theta$    & \verb+\theta+ \\
$\delta$      & \verb+\delta+      & $\vartheta$ & \verb+\vartheta+ \\
$\epsilon$    & \verb+\epsilon+    & $\iota$     & \verb+\iota+ \\
$\varepsilon$ & \verb+\varepsilon+ & $\kappa$    & \verb+\kappa+ \\
\hline
\end{tabular}
\end{table}

Use the \verb+thebibliography+ environment for the references.  BibTeX users may
use the provided BibTeX style file DeGruyter.bst.

%% BibTeX support
\bibliographystyle{DeGruyter}
\bibliography{sample}

\end{document}